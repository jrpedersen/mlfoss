\notocchapter{Introduction}
\label{chap:intro}
	%\caption{The Meatmaster II.}
	%\label{label_ex2_nofod}
%\end{marginfigure}
\sidefigure{Wilhelm Röntgen, from \cite{WilhelmConradRontgen2020}}{	\includegraphics[width=\linewidth]{figures/chapter1/Wilhelm_Röntgen_by_Nicola_Perscheid_1915b.jpg}}[3]
%Context
% X-ray absiometry
%\subsubsection{Context x-rays}
% X-ray introduction
%Since the discovery of x-rays in 1895 by Wilhelm Röntgen, x-rays have found uses in countless places. 
%Since the discovery of x-rays in 1895 by Wilhelm Röntgen,
Wilhelm Röntgen discovered x-rays in 1895 as a penetrating electro-magnetic radiation.
The ability to penetrate matter made it possible to take pictures of the insides of solid objects, and this found use almost immediately in the medical industry.
% What is it?
%This ability to see through otherwise solid objects found use almost immediately in the medicinal industry.%\sidenote{With the first medical image taken of his wife's hand a month later.}
This resulted in Wilhelm recieving the first Nobel prize in physics in 1901 \cite{NobelPrizePhysics}.
%The ability to see through otherwise solid objects is used everywhere from airport security to the medicinal industry to the food processing one.
% Where has it been used?
Today, x-rays have numerous applications: From airport security, in the medical industry to food processing.
%This has found use everywhere from airport security to the medicinal industry to the food processing one.
% Food industry & Non-destructive
In the food processing industry, the ability to do non-destructive inspection of food is an incredibly cost effective way to make quality control.
Contrary to a destructive procedure using statistical sampling to test, with x-rays it is possible to scan objects non-destructively. 
For this reason, x-ray technologies has been adopted in various food lines in factories.\cite{haffXrayDetectionDefects2008}%\todo{CItations or examples?}

% Optics and computer vision
%\subsubsection{Context CV and ML}
% Test all -> lot of data
%With the ability to test all the food in a production line enormous amounts of data are generated.
The ability to inspect all the food in a production line generates enormous amounts of data.
% Lot of data -> automatization
To make proper use of the data, the analysis needs to be automatic.
%General to optimics, and also for x-rays, is the production of images. 
% Automatization to computer vision
%Automating the treatment images, in the case of x-ray imaging, is the field of computer vision.
The field of computer vision concerns the automated analysis of images.
% ML revolution in computervision
The past years, the field of computer vision has undergone a revolution with the introduction of deep learning \cite{lecunDeepLearning2015}.
The development of deep learning has been driven by companies like Google and Facebook, who are pushing the boundaries of what is possible in many areas of application.
%been dominated by machine learning as companies like Google and Facebook are pushing the boundaries for what is possible.
% Deep learning briefly
With deep learning, it is possible to take advantage of large datasets to create models of greater size than before in order to get extremely good performance.
% Deep learning made possible by speed up in hardware.
%This is enabled by the increased capabilities of hardware.%\todo{References for this paragraph.}

% Foss Introduction
% Foss mission
\sidefigure{Meat Master II, from \cite{fossMeatMasterIIOne}}{	\includegraphics[width=11.5pc]{images/MeatMaster_II_1000x1000.jpg}}[3]
%\subsubsection{Context FOSS}
A company selling optical instruments to use in food production is FOSS, and this piece of work has been birthed in a collaboration with them. 
FOSS sells the Meat Master II \cite{100MeatAnalysis} which is used in-line in meat production to scan meat. 
The Meat Master II uses x-rays to primarily determine the fat composition of meats. 
As a secondary task it scans the meat for foreign objects.
In meat possible foreign elements could be naturally appearing matter such as bone.
%\todo{Is these technically foreign objects?}
More troublesome, is the chance of plastic or metal, which could appear from unfortunate accidents, or perhaps more worrisome, in the form of tampering of the product \cite{nov28HalifaxPoliceInvestigating2016}.
% or unfortunate artefacts of the production such as metal or plastic.
%The physics behind how the x-ray images are taken, are for now not \textit{directly} relevant for the work I have made.\\
%The images/data belongs to FOSS, and it consist of x-ray images produced with their Meat Master 2 \todo{Ref - meat measter 2} machine. The main purpose of this machine is determination of fat percentage in meat, but while doing this it also checks for foreign objects.

%Task
%\subsubsection{Task}
% Current methods or one way to implement an algorithm
The current methods used in the Meat Master II for foreign object detection are threshold-based using computer vision.
With their threshold-based algorithm, pieces of metal are the easiest to find, small pieces of bone is almost undetectable, and thin slices of plastic are impossible.
% Problems with threshold methods
% Relate to the current aim of the thesis.
The aim of this thesis has been to explore machine learning, and more specifically deep learning, to do foreign object detection.
The expectation is that it should be possible to implement an algorithm that can detect foreign objects, of metal and bone.
% Perspectivize a little
This takes starting point in data from the Meat Master II, but the methods explored here might have more general applications.
% Sell your own work
%If one buys into the story of machine learning replacing traditional computer vision methods the best case for thesis would be to introduce this shift in methodology here.\sidenote{SPOILER ALERT: This overarching goal is in no way completed with this work.}
%This thesis attempts to analyse the use of machine learning for foreign object detection in x-ray images of food. 
%\todo{Deep learning yaalalala}.
%Central to all of machine learning is the data. Without data no learning. In general data can take almost any form, but for this project I have mainly been working with images.
%Object
% Clarifies what the document covers:
% Covers these contributions:
% Trained a ConvNet
%%%( Deep learning cool)
% Labelled FOSS data with some programs.
% Artificial foreign objects
% Overview of Adam optimizers
% Thorough analysis of hyper parameters.
%%% Shoudl have been bayesian but what evs.
%In broad terms these can be grouped as a data part, a learning part, and finally a model part.\\
%\subsubsection{Objects}
%\sidedef{Part I}{}{

This thesis is divided into two parts: X-ray Physics and Setup, followed by Foreign Object Detection with Deep Learning.
%The first part sets the stage for the introduction of deep learning. 
\sidedef{Part I}{}{
%I introduce X-ray physics (\chapref{chap:x-ray}) both the interactions of interest and the source and detector necessary to use it practically.
X-ray Physics and Setup introduces the physics of x-rays and the data made available by FOSS.
%To begin with, the physics of x-rays are introduced (\chapref{chap:x-ray}). 
This includes the electromagnetic interactions of interest and the source and detector necessary for x-ray imaging (\chapref{chap:x-ray}).
Then the images that constitute the available data are presented and the labelling of the foreign elements is shown (\chapref{chap:datapres}).
%I present the images that constitute the data I have had to work with and explains how I have labelled the data (\chapref{chap:datapres}).
}
%\todo{Introduce acronyms form here?}
\sidedef{Part II}{}{
Foreign Object Detection with Deep Learning follows the four steps of a deep learning algorithm as laid out by the main textbook cited \cite[see][sec 5.10]{goodfellowDeepLearning2016}:
%Following the recipe laid out in \cite[see][sec 5.10]{goodfellowDeepLearning2016} I present the process of using deep learning in four discrete steps: 
\textit{"... combine a specification of a dataset, a cost function, an optimization procedure and a model"}. 
First, the data processing leading to the final dataset is specified.
%I specify the final dataset I have used (\chapref{chap:dataprocs}). 
This entailed implementing a sliding window algorithm to preprocess the data into a format better suited for the model.
Furthermore, simple data augmentation transformations, along with a more complex one of my own design which introduces synthetic/artificial foreign objects are described (\chapref{chap:dataprocs}).
%I have implemented a sliding window algorithm to preprocess the data into a format better suited for my model. Furthermore, I present simple data augmentation transformations, along with a more complex home-brewed one which introduces synthetic/artificial foreign objects.
Then the cost function and optimizers used are presented. 
The cross-entropy of the negative log-likelihood was used for cost function. 
For optimizers, \ac{SGD} with momentum along with variations of \ac{Adam} was used (\chapref{chap:learning}).
%Then I present the cost function and optimizers I have been using (\chapref{chap:learning}). For cost function I have been using Cross entropy. As for optimizers, I have looked at SGD with/ momentum along with the whole family of Adams .
%I design a model to train which consists of various components which are chosen by optimization (\chapref{chap:nn}).
The design of the deep learning model is presented next (\chapref{chap:nn}).
This concludes the four steps describing the deep learning algorithm, and left is only the evaluation of the approach.
The results are presented, both for windows and full pictures.
Furthermore, the performance of the model outside its training regime is explored (\chapref{chap:results}). 
%Furthermore, I try to explore the performance of the model outside its' training regime (\chapref{chap:results}). 
%Drawing it all to a close I discuss and conclude in \chapref{chap:conclusion}.
Drawing it all to a close, the discussion and conclusion are last (\chapref{chap:conclusion}).
}

% Introduc Pytorch
For deep learning I used the PyTorch library.
All of the code, for both the thesis and the models, should be available at \url{https://github.com/jrpedersen/mlfoss}.